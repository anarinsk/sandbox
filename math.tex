\documentclass[]{article}
\usepackage{amssymb,amsmath}
\usepackage{ifxetex,ifluatex}
\ifxetex
  \usepackage{fontspec,xltxtra,xunicode}
  \defaultfontfeatures{Mapping=tex-text,Scale=MatchLowercase}
\else
  \ifluatex
    \usepackage{fontspec}
    \defaultfontfeatures{Mapping=tex-text,Scale=MatchLowercase}
  \else
    \usepackage[utf8]{inputenc}
  \fi
\fi
\usepackage{color}
\usepackage{fancyvrb}
\DefineShortVerb[commandchars=\\\{\}]{\|}
\DefineVerbatimEnvironment{Highlighting}{Verbatim}{commandchars=\\\{\}}
% Add ',fontsize=\small' for more characters per line
\newenvironment{Shaded}{}{}
\newcommand{\KeywordTok}[1]{\textcolor[rgb]{0.00,0.44,0.13}{\textbf{{#1}}}}
\newcommand{\DataTypeTok}[1]{\textcolor[rgb]{0.56,0.13,0.00}{{#1}}}
\newcommand{\DecValTok}[1]{\textcolor[rgb]{0.25,0.63,0.44}{{#1}}}
\newcommand{\BaseNTok}[1]{\textcolor[rgb]{0.25,0.63,0.44}{{#1}}}
\newcommand{\FloatTok}[1]{\textcolor[rgb]{0.25,0.63,0.44}{{#1}}}
\newcommand{\CharTok}[1]{\textcolor[rgb]{0.25,0.44,0.63}{{#1}}}
\newcommand{\StringTok}[1]{\textcolor[rgb]{0.25,0.44,0.63}{{#1}}}
\newcommand{\CommentTok}[1]{\textcolor[rgb]{0.38,0.63,0.69}{\textit{{#1}}}}
\newcommand{\OtherTok}[1]{\textcolor[rgb]{0.00,0.44,0.13}{{#1}}}
\newcommand{\AlertTok}[1]{\textcolor[rgb]{1.00,0.00,0.00}{\textbf{{#1}}}}
\newcommand{\FunctionTok}[1]{\textcolor[rgb]{0.02,0.16,0.49}{{#1}}}
\newcommand{\RegionMarkerTok}[1]{{#1}}
\newcommand{\ErrorTok}[1]{\textcolor[rgb]{1.00,0.00,0.00}{\textbf{{#1}}}}
\newcommand{\NormalTok}[1]{{#1}}
\ifxetex
  \usepackage[setpagesize=false, % page size defined by xetex
              unicode=false, % unicode breaks when used with xetex
              xetex,
              colorlinks=true,
              linkcolor=blue]{hyperref}
\else
  \usepackage[unicode=true,
              colorlinks=true,
              linkcolor=blue]{hyperref}
\fi
\hypersetup{breaklinks=true, pdfborder={0 0 0}}
\setlength{\parindent}{0pt}
\setlength{\parskip}{6pt plus 2pt minus 1pt}
\setlength{\emergencystretch}{3em}  % prevent overfull lines
\setcounter{secnumdepth}{0}


\begin{document}

Add mathjax script right into the markdown file like so:

\begin{Shaded}
\begin{Highlighting}[]
\CommentTok{<!-- Equations using MathJax -->}
\KeywordTok{<script}\OtherTok{ type=}\StringTok{"text/javascript"}\OtherTok{ src=}\StringTok{"http://cdn.mathjax.org/mathjax/latest/MathJax.js?config=TeX-AMS-MML_HTMLorMML"}\KeywordTok{></script>}
\KeywordTok{<script}\OtherTok{ type=}\StringTok{"text/x-mathjax-config"}\KeywordTok{>} \KeywordTok{MathJax.Hub}\NormalTok{.}\FunctionTok{Config}\NormalTok{(\{ }\DataTypeTok{TeX}\NormalTok{: \{ }\DataTypeTok{equationNumbers}\NormalTok{: \{}\DataTypeTok{autoNumber}\NormalTok{: }\StringTok{"all"}\NormalTok{\} \} \});       }\KeywordTok{</script>}
\end{Highlighting}
\end{Shaded}
Add knitr block into the markdown file.

\begin{Shaded}
\begin{Highlighting}[]
\DecValTok{1} \NormalTok{+ }\DecValTok{1}
\end{Highlighting}
\end{Shaded}
\begin{verbatim}
## [1] 2
\end{verbatim}
Display equations by delimiting with
\texttt{\textless{}div\textgreater{}\$\$}

\[  J_\alpha(x) = \sum\limits_{m=0}^\infty \frac{(-1)^m}{m! \, \Gamma(m + \alpha + 1)}{\left({\frac{x}{2}}\right)}^{2 m + \alpha} \]
And inline equations with \texttt{\textless{}span\textgreater{}\$}, like
$e = mc^2$

\subsubsection{Now the following works:}

\begin{itemize}
\item
  Pandoc generated latex: \texttt{pandoc -s math.md -o math.tex}.\\
\item
  The Markdown file will be converted to valid html by almost any
  markdown converter, which ignores the div/span elements. The mathjax
  script added in the top will cause these to display correctly. (Note
  this works for the inline equation too, even though
  \href{http://www.mathjax.org/docs/2.0/start.html\#tex-and-latex-input}{mathjax
  says it shouldn't}?)
\item
  To get pandoc to run in this mode without touching the math syntax,
  use \texttt{-{}-strict} option:
  \texttt{pandoc -s -{}-strict math.md -o math.html}
\item
  Pandoc can handle the mathjax rendering itself, since without
  \texttt{strict} enabled, it reads inside the div/span elements and
  finds math syntax it recognizes. as does
  \texttt{pandoc -s -{}-mathjax math.md -o math3.html}. Likewise for
  mathml: \texttt{pandoc -s -{}-mathml -o math4.html}.
\item
  Simple html works too, though has no rendering engine:
  \texttt{pandoc -s math.md -o math2.html} so the result is not
  displayed properly.
\end{itemize}
\subsubsection{See results}

\begin{itemize}
\item
  \href{https://github.com/cboettig/sandbox/blob/master/math.md}{math.md}
\item
  \href{https://github.com/cboettig/sandbox/blob/master/math.tex}{math.tex}
\item
  Pandoc strict (or most markdown intepreters):
  \href{http://cboettig.github.com/sandbox/math3.html}{math.html}
\item
  Pandoc mathjax:
  \href{http://cboettig.github.com/sandbox/math3.html}{math3.html}
\item
  mathml:
  \href{http://cboettig.github.com/sandbox/math4.html}{math4.html}
\item
  Pandoc's plain html:
  \href{http://cboettig.github.com/sandbox/math2.html}{math2.html}
\item
  Giving jekyll a copy of math.md, it creates
  \href{http://cboettig.github.com/sandbox/math-jekyll.html}{math-jekyll.html}
\end{itemize}

\end{document}
